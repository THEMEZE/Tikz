\documentclass[tikz, border=2mm]{standalone}


\begin{document}
\usetikzlibrary{perspective}
\newcommand\simplecuboid[3]{%
  \fill[gray!80!white] (tpp cs:x=0,y=0,z=#3)
    -- (tpp cs:x=0,y=#2,z=#3)
    -- (tpp cs:x=#1,y=#2,z=#3)
    -- (tpp cs:x=#1,y=0,z=#3) -- cycle;
  \fill[gray]  (tpp cs:x=0,y=0,z=0)
    -- (tpp cs:x=0,y=0,z=#3)
    -- (tpp cs:x=0,y=#2,z=#3)
    -- (tpp cs:x=0,y=#2,z=0) -- cycle;
  \fill[gray!50!white] (tpp cs:x=0,y=0,z=0)
    -- (tpp cs:x=0,y=0,z=#3)
    -- (tpp cs:x=#1,y=0,z=#3)
    -- (tpp cs:x=#1,y=0,z=0) -- cycle;}
\newcommand{\simpleaxes}[3]{%
  \draw[->] (-0.5,0,0) -- (#1,0,0) node[pos=1.1]{x};
  \draw[->] (0,-0.5,0) -- (0,#2,0) node[pos=1.1]{y};
  \draw[->] (0,0,-0.5) -- (0,0,#3) node[pos=1.1]{z};}




\newcommand{\test}[9]{
%Cube ( 0 < x < #1 , 0 < y < #2 , 0 < z < #3 )
%perspective={p = {(#4,0,0)},q = {(0,#5,0)},r = {(0,0,#6)}}
%3d view={#7}{#8}
%scale = 1/#9 
\begin{scope}[
	3d view={#7}{#8},
  	perspective={
    p = {(#4,0,0)},
    q = {(0,#5,0)},
    r = {(0,0,#6)}},
  	scale=1,
  	vanishing point/.style={fill,circle,inner sep=2pt}]
	]
	
	\simplecuboid{#1}{#2}{#3}
  	\simpleaxes{#9}{#9}{#9}

  	\node[vanishing point,label = right:p] (p) at (#4,0,0){};
  	\node[vanishing point,label = left:q] (q) at (0,#5,0){};
  	\node[vanishing point,label = above:r] (r) at (0,0,#6){};

  	\begin{scope}[dotted]
    	\foreach \y in {0,#2}{
      	\foreach \z in {0,#3}{
        \draw (tpp cs:x=0,y=\y,z=\z) -- (p.center);}}
    \foreach \x in {0,#1}{
      	\foreach \z in {0,#3}{
        \draw (tpp cs:x=\x,y=0,z=\z) -- (q.center);}}
    \foreach \x in {0,#1}{
      	\foreach \y in {0,#2}{
        \draw (tpp cs:x=\x,y=\y,z=0) -- (r.center);}}
  \end{scope}
	
\end{scope}

}

\begin{tikzpicture}
	%\filldraw [shift={(0,0)} ,line width=0.5ex,rounded corners = 0ex,color=red,fill=red!40 , ] (-5,-5) rectangle (5,  5);
	\clip (-5,-5) rectangle (5,  5);
	\test{1}{1}{1}{100}{100}{100}{0}{180}{2}		
\end{tikzpicture}



\end{document}
